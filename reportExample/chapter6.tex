\chapter{Testing and evaluation}
\label{cha:testing}

The initial plan was to have a total of 10 participants to take part in a 2-hour testing process. However, some people dropped out while other people expressed interest, so the final number of participants was 13. All the participants were university students. The testing took place in week 4 of second semester (17th - 20th of February).

At the moment of user testing, the game was finished with the exception of the car model and colour choice.

In this chapter, we are going to discuss the results of the user testing. The used questionnaire can be found in appendix \ref{appendix:userStudy}. 

For the purposes of testing the projects, the participants were firstly asked to play the game in keyboard mode. Following this, they were asked to fill in a NASA TLX (Task Load Index) test so that we would have a measurement of how demanding this task was for the user. Next, the Emotiv EPOC headset was fit on the partipant's head and the training process would begin. After feeling confident enough for the first action (push), the user could see it working in the actual game play. We insisted on properly training 2 actions: push - for acceleration - and left - for a left turn. The reason why left was necessary is because the track does a left turn on a slightly elevated terrain. Less than 5 users have probably had enough time to train the other 2 states (pull and right) as well. Again, after the 2 hours were almost up, the participant was asked to take another NASA TLX test, this time taking into account how demanding the process was for Cognitiv game play. Finally, the participant was asked to fill in the user study questionnaire in appendix \ref{appendix:userStudy}. 

The following sections will describe the results of the user testing. In section \ref{section:implemented}, we will highlight what was changed in CogniDriver following the participants answer, in the period of 2 weeks left before the project demonstrations period begun.

\section{Observations}

Some of the participants displayed better skill in training the headset if the testing was done in the morning. This is compared to evenings, when some of the partipants complained about being tired and not able to concentrate properly.

\section{NASA TLX results}
The test was taken using the \cite{nasatlx} online version developed by Keith Vertanen. It produces the analysis of the input data for each person at the end of the test. 

For keyboard mode, the overall mean difficulty was 51.611\% with a standard deviation of 17.494\%.

For Cognitiv mode, the overall mean difficulty was 90.571\% with a standard deviation of 13.964\%.

This tells us, the participants have found the game much easier to play in keyboard mode compared to Cognitiv mode.

\section{Analysis of user questionnaire answers}
The user questionnaire aimed to measure things such as the comfort of wearing the headset and how demanding the training and playing were, especially in Cognitiv mode.

The mean of the comfort of wearing the headset was 5.692 out of 10 with an approximately equal spread of scores. Most of the participants have started feeling the discomfort after 20-40 minutes of wearing the headset. 

For the questions regarding the demand of the process, the participants' scores were quite unanimous. The training demand obtained a score of 4.23/10, the play demand obtained a score of 3.153/10 and the ease demand obtained 2.153/10. The similarity between the keyboard play and Cognitiv play was of 2.307/10 suggesting the two modes are not similar.

\section{Good points about the game}

\section{Bad points about the game}

\section{What should be improved?}

\section{What has been implemented?}
\label{section:implemented}

