\chapter{Introduction}
\label{cha:intro}

\section{Applications}

Using BCIs (Brain Computer Interfaces) to control an application is of great use to persons with motor disabilities for whom the use of keyboard or mouse devices might not be convenient. In addition,  BCIs provide a great way to exercise brain functions and improve concentration. \cite{adhd} shows how the use of a BCI platform may improve the attention and concentration of ADHD patients. \cite{manipulator} shows a way of remote-controlling an industrial manipulator through the use of an EEG system. \cite{remotecontroller} describes the use of a SSVEP type system with 48 LEDs to match key-presses of a television remote-controller. SSVEP (Steady State Visually Evoked Potential) is a signal which is the brain's natural response to visual stimulation at specific frequencies. \cite{exhibition} uses a BCI to study the performances and preferences of 21 participants while playing a `Star Wars' themed game. For more examples of real life uses of brain computer interfaces, see Section \ref{sec:currentApplication}.

\section{Aims and objectives}

The aim of this project was to learn about the pros and cons of BCIs, how these interact with the computer and reach a conclusion about how they could be used in the future. Because the Emotiv EPOC Control Panel (Chapter \ref{cha:epoc}) allows only up to 4 actions to be recognised at any one time, the simplest choice of a game was that of a car driving one since the player can also observe the car moving in the dictated direction and this may ease the activation of an action through the visual feedback.

The key objectives of the project were to:
\begin{itemize}
	\item Allow the car to move in each one of the four directions: forward, reverse, left, right through keyboard use; 
	\item Reach access to the interpreted data of the developer's SDK;
	\item Allow players to train new profiles inside the game;
	\item Provide an array of cars and colours that the user can select from.
\end{itemize}

\section{Report outline}

The report begins with describing the used tools and provides some information about the way the brain works in Chapter \ref{cha:background}. Next, Chapter \ref{cha:epoc} describes how the Emotiv EPOC works and how it is organised. In this chapter, the report also contains an overview of some other applications developed with this technology.

Chapter \ref{cha:design} focuses on the description of the design of this project, while Chapter \ref{cha:implementation} oversees the implementation details. Chapter \ref{cha:testing} describes the performed testing and Appendix \ref{appendix:userStudy} shows the user questionnaire used during the user trials. Finally, Chapter \ref{cha:conclusions} contains the conclusions of this project. Appendix \ref{appendix:3dobjects} provides the sources of the external 3D objects used in the project.