\chapter{Introduction}
\label{cha:intro}

\section{Applications}

Using BCI (Brain Computer Interfaces) to control an application is of great use to persons with motor disabilities whose use of keyboard or mouse devices might not be convenient. In addition,  BCIs provide a great way to exercising brain functions and improving concentration.

\section{Aims and objectives}

The aim of this project was to learn about the ... and falls of BCIs, how these interact with the computer and reach a conclusion about how they could be used in the future. Because the Emotiv EPOC Control Panel allows only up to 4 actions to be recognised at any one time, the simplest choice of a game was that of a car driving one since you can also observe the car moving in the dictated direction which can ease the activation of an action through the visual feedback.

The key objectives of the project were to:
\begin{itemize}
	\item Allow the car to move in each one of the four directions: forward, back, left, right through keyboard use; 
	\item Reach access to the interpreted data of the developers' SDK;
	\item Allow players to train new profiles inside the game;
	\item Provide an array of cars and colours that the user can select from.
\end{itemize}

\section{Report outline}

The report begins with describing the used tools and provides some information about the way the brain works in Chapter 2. Next, Chapter 3 describes how the Emotiv EPOC works and how it is organised. In this chapter, the report also contains a quick look over some other applications developed with this technology.

Chapter 4 focuses on the description of the design of this project, while Chapter 5 oversees the implementation details. Chapter 6 reviews the obtained results, while Chapter 7 described the performed testing. Appendix 1 shows the user questionnaire used during the user trials. Finally, Chapter 8 contains the conclusions of this project.